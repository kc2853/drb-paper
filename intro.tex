%!TEX root = main.tex

\section{Introduction}
\label{sec:intro}

Public, trustworthy randomness has been a goal for millennia, dating at least to the earliest known use of dice around 3000 BCE. Today, public randomness is crucial to applications including gambling and lotteries~\cite{bonneau2015bitcoin}, electronic voting~\cite{adida2008helios}, selecting parameters for cryptographic protocols~\cite{baigneres2015trap, lenstra2015random}, leader election in proof-of-stake protocols~\cite{gilad2017algorand, kiayias2017ouroboros}, and blockchain sharding~\cite{al2017chainspace, kokoris2018omniledger}.

The concept of a \emph{randomness beacon} was first formalized by Rabin~\cite{rabin1983Rabin} to describe an ideal service that regularly emits fresh random values that no party can manipulate or predict. Because no such ideal beacon exists, various protocols are used to approximate this beacon functionality for practical use.
%solutions ranging from centralized approaches (relying on a single source or organization) to distributed approaches (decentralizing the randomness generation among a set of nodes) are used to approximate it.

\textbf{Centralized Beacons.} Relying on a trusted third party like NIST~\cite{fischer2011public,kelsey2019reference} or random.org~\cite{haahr2010random} might be the simplest way to realize a beacon. It carries drawbacks typically associated with centralized services, such as the risk of compromise or misbehavior and the inability of the end user to verify the security of the beacon.
In particular, it is straightforward to design a malicious beacon that outputs statistically random values which are predictable given a trapdoor. For example, given a semantically secure encryption scheme the underhanded beacon can simply use a secret key to encrypt a counter in each interval.
Security of the underlying encryption scheme guarantees this is indistinguishable from random without access to the key, but completely predictable given the key.
%For this reason, it is strongly desirable to construct a beacon with no trusted party or central point of failure.

\textbf{Implicit Beacons.} Another approach is to construct a beacon using publicly available implicit sources of entropy such as stock market data~\cite{clark2010use} or proof-of-work (PoW) blockchains like Bitcoin~\cite{nakamoto2008bitcoin, bentov2016bitcoin, bonneau2015bitcoin, han2020randchain}. These entropy sources are potentially vulnerable to malicious insiders (e.g. high-frequency traders making unnatural trades to fix stock prices, financial exchanges blocking trades or reporting incorrect data, miners that can withhold blocks or choose between colliding blocks, etc.). These beacons are plausibly secure and low-cost in practice, but they still lack formal models of security. As a result, while we consider these important targets for future research, we will not discuss them in detail in this work.
%The entropy of these sources can also be difficult to precisely measure. 

\textbf{Distributed Randomness Beacons.}
A natural approach to reduce trust in a centralized beacon is a multi-party \textit{distributed randomness beacon} (DRB). DRB protocols are designed to remain secure and live despite some fraction of malicious participants. % and a potentially untrusted network. %We call a beacon realized in this manner a \textit{distributed randomness beacon} (DRB). 
DRB protocols are typically \epoch-based, producing fresh random output in each \epoch.

%\textbf{Contribution.} 
The goal of this paper is to systematize current research on DRBs. We propose a general framework encompassing all DRB protocols in the landscape. To aid comparison and discussion of properties, we provide an overview of these protocols along with the cryptographic building blocks used to construct them. We identify two key components of DRB design: selection of entropy providers and beacon output generation, which can be decoupled from each other. Enabling a more holistic analysis of a DRB as a result, we also provide new insights and discussion on potential attack vectors, countermeasures, and techniques that lead to better scalability.

\textbf{Paper organization.} We begin with preliminaries including our system model, a strawman DRB under perfect synchrony (an ideal assumption), commit-reveal~\cite{blum1983coin}, and the definition of an ideal DRB in Section~\ref{section:preliminaries}. Section~\ref{section:delay} introduces protocols using \textit{delay functions} (verifiable delay functions~\cite{boneh2018verifiable} and timed commitments~\cite{boneh2000timed}), which offer the best fault tolerance (dishonest majority) and simplicity, assuming secure delay functions can be implemented in practice. In Section~\ref{section:commit-reveal-punish} to~\ref{section:dvrf}, we introduce non-delay-based DRB protocols categorized by the number of nodes contributing \textit{marginal entropy} (i.e. per-\epoch randomness that is independently generated at a node level) in each \epoch. Sections~\ref{section:commit-reveal-punish} and~\ref{section:commit-reveal-recover} review protocols in which all nodes contribute marginal entropy. These protocols vary in mechanisms used to recover from faulty nodes, including financial punishment~\cite{youcai2017randao, david2020economically}, threshold secret sharing~\cite{schoenmakers1999simple, cascudo2017scrape}, and threshold encryption~\cite{desmedt1990Threshold}. Section~\ref{section:committee-based} covers committee-based protocols in which each \epoch includes an extra committee selection step, after which only a committee (subset) of nodes contributes marginal entropy. These protocols are more complex but can offer greater communication efficiency with large numbers of nodes. Section~\ref{section:dvrf} covers pseudorandom protocols that do not require any marginal entropy; these protocols can be highly efficient but have no mechanism to recover from compromise.
% We conclude with discussion and comparisons in Sections~\ref{section:discussion}--\ref{section:conclusion}, including a comparison of all studied protocols in Table~\ref{table:comparison}.
We conclude with discussion and comparisons in Sections~\ref{section:discussion}--\ref{section:conclusion} and Table~\ref{table:comparison}.
