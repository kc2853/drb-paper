%!TEX root = main.tex
Motivated and inspired by the emergence of blockchains, many new protocols have recently been proposed for generating publicly verifiable randomness in a distributed yet secure fashion. These protocols work under different setups and assumptions, use various cryptographic tools, and entail unique trade-offs and characteristics. In this paper, we systematize the design of distributed randomness beacons (DRBs) as well as the cryptographic building blocks they rely on.
We evaluate protocols on two key security properties, unbiasability and unpredictability, and discuss common attack vectors for predicting or biasing the beacon output and the countermeasures employed by protocols. We also compare protocols by communication and computational efficiency.
%We start by delving into protocols where every node plays an active role contributing to the final randomness. Next, we provide an overview of techniques used to select and generate the randomness with only a subset of active nodes to improve scalability. However, this may come at the cost of diminished quality of randomness or vulnerability to biasing by an adversary.
%We then discuss some attack vectors for predicting or biasing the beacon and the countermeasures employed by protocols.
Finally, we provide insights on the applicability of different protocols in various deployment scenarios and highlight possible directions for further research.
