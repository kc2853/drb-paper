%!TEX root = main.tex

\section{Committee-Based Protocols}
\label{section:committee-based}
While all aforementioned commit-reveal variants (Sections~\ref{subsection:modifying-commit-reveal},~\ref{section:commit-reveal-punish}, and~\ref{section:commit-reveal-recover}) include every node in each entropy-providing committee $\committee_\epochv$, this has a scalability problem.
Requiring communication of marginal entropy by all nodes is inefficient with large numbers of participants, and hence a natural optimization is to use a smaller subset of nodes $\committee_\epochv$ in each \epoch to contribute entropy (i.e. reduce $|\committee_\epochv|$).

In this section, we consider DRBs that are committee-based, with $\committee_\epochv$ such that $1 \leq |\committee_\epochv| < n$. Committee-based protocols proceed in two steps: \textit{committee selection} and \textit{beacon output generation}. As the names suggest, $\committee_\epochv$ is agreed upon during committee selection while the beacon output $\drboutput_\epochv$ is generated and agreed upon during beacon output generation.
%Provided below is a summary of ten preexisting protocols, and we offer intuition (refer to corresponding citations for full details) on interpreting them under our simple framework.
We observe that committee selection and beacon output generation are, at least theoretically, modular such that subprotocols can be independently chosen for the two components. We visualize these two dimensions of committee-based DRBs in Table~\ref{table:committee-based}. We also observe that the protocols introduced so far (e.g. commit-reveal-recover) can be used as a module in a larger committee-based protocol, with the chosen committee executing the chosen protocol in each \epoch.

\subsection{Step 1. Committee Selection}
The first step of a committee-based DRB involves selecting $\committee_\epochv$ in a way agreeable by all nodes. We classify committee selection mechanisms into two: public and private.

\subsubsection{Public Committee Selection}
\label{subsubsection:public-committee-selection}
In a \textit{public committee selection}, only public information is needed to derive $\committee_\epochv$.\\

\noindent\textbf{Round-Robin (RR).} A first example is \textit{round-robin} (RR), in which nodes simply take predetermined turns being selected. While RR can work with committees of any size, typically RR is used to select a committee of size one (i.e. a leader) corresponding to node $i \equiv \epochv \pmod n$. Protocols like BRandPiper~\cite{bhat2020randpiper} (in which the \epoch leader is the only active entropy provider) adopt RR as their leader selection mechanism due to its innate fairness property~\cite{azouvi2018winning} (also known as chain quality~\cite{garay2015bitcoin} in the blockchain context) where all nodes, by RR's definition, take equal leadership.\\
%Looking ahead, RR is vulnerable against an adaptive adversary, which can predict in advance which nodes to corrupt to control a committee in a given round. Thus, it is important that RR not assume any committee (or leader) is honest.\\

\noindent\textbf{Random Selection (RS).} A second example is \textit{random selection} (RS), which uses some public randomness (most commonly the last beacon output $\drboutput_{\epochv - 1}$) to derive $\committee_\epochv$.
%While RR is simple and deterministic, RS is randomized.
In HydRand~\cite{schindler2020hydrand} and GRandPiper~\cite{bhat2020randpiper}, for instance, $\committee_\epochv$ consists of a node $i \equiv \drboutput_{\epochv - 1} \pmod{\tilde{n}}$ where $\tilde{n}$ is the number of eligible nodes. Similar is the process in Ouroboros~\cite{kiayias2017ouroboros} called follow-the-satoshi~\cite{bentov2014proof,kiayias2017ouroboros}, which can output more than one node in $\committee_\epochv$.

Randomized selection means that some nodes may, in theory, never be selected and therefore may never be able to provide entropy contribution. A more serious concern is that an adversary can attempt to bias, via grinding attack, $\drboutput_\epochv$ in order to bias $\committee_{\epochv + 1}$ (which can bias $\drboutput_{\epochv + 1}$). In the worst case, this can lead to a vicious cycle in which an adversary controlling enough nodes on the current committee to manipulate the beacon output can ensure it will also control enough nodes on the next committee, and so on ad infinitum.

This is not an issue in RR, as its committee selection is deterministic and independent of the preceding beacon output. Nonetheless, a tradeoff of RR is that DoS attack becomes indefinitely possible (for all \epochs $\epochv$ for $\epochv > \tilde{\epochv}$ given $\drboutput_{\tilde{\epochv}}$) since each committee is publicly known in advance. All in all, RR gains unbiasability (due to determinism) at the cost of indefinite DoS attack while RS benefits from less DoS attack, i.e. that only for \epoch $\epochv + 1$ (due to randomization given $\drboutput_\epochv$), at the cost of grinding attack.\\

\noindent\textbf{Leader-Based Selection (LS).} A third example, \textit{leader-based selection} (LS) is a hybrid method that exhibits both determinism and randomization. It runs in two steps: the first step involves electing an \epoch leader (either by RR or RS) while the second involves selection of $\committee_\epochv$ by the elected leader. It is in this way that the mechanism is deterministic from the leader's perspective while randomized from that of others.

One approach to limit the power delegated to the leader is that $|\committee_\epochv|$ needs to be greater than $t$ so that a malicious leader wouldn't be able to choose $\committee_\epochv$ maliciously. RandHound~\cite{syta2017scalable}, SPURT~\cite{das2021spurt}, and OptRand~\cite{bhat2022optrand} demonstrate such LS.
\begin{itemize}
\item RandHound. As instantiated in RandHerd~\cite{syta2017scalable}, RandHound's leader election (i.e. via RS as the first step of LS) involves a public lottery where each node generates a lottery ticket $H(C \mathbin\Vert pk_i)$ given a public configuration parameter $C$ (assuming its randomness) such that node $\argmin_{i} H(C \mathbin\Vert pk_i)$ becomes the leader (originally called client). In the second step of LS, RandHound adopts a form of sharding (involving PVSS groups). The leader selects more than a threshold number of nodes in each shard (PVSS group), guaranteeing a threshold number of entropy providers across all shards.
\item SPURT and OptRand. Unlike RandHound, SPURT and OptRand adopt RR as its first part of LS such that nodes simply take turns being an \epoch leader. Then the leader chooses $\committee_\epochv$ based on received encrypted messages.
\end{itemize}

Given an underlying DRB that utilizes the concept of a leader as an orchestrator of communication among nodes, LS is a natural choice to committee selection, as a leader helps mitigate the protocol's communication cost overall.

\subsubsection{Private Committee Selection}
\label{subsubsection:private-committee-selection}
In a \textit{private committee selection}, also known as \textit{private lottery}, each node needs to input some private information (e.g. secret key) in order to check whether or not it has been selected into $\committee_\epochv$ (i.e. has won the lottery). The general formulation of a private lottery can be given by
\[
f_{priv}(\cdot) < target
\]
where $f_{priv}(\cdot)$ is a lottery function (i.e. pseudorandom function) that takes some private input $priv$ and $target$ denotes the lottery's ``difficulty level'' (a la Proof of Work difficulty) such that $target$ can be adjusted to make the lottery arbitrarily easy or hard to win.

Each node calculates $f_{priv}(\cdot)$ and checks if the above inequality is satisfied, in which case it ``wins'' the lottery and becomes an entropy provider. As an adversary can perform a grinding attack by trying many values of $priv$ until a desirable function output is achieved, one crucial requirement is that $priv$ should be provably committed in the past and thus be ungrindable at the time of computation of $f_{priv}(\cdot)$.

A prime example of a private lottery is a VRF-based lottery. Most notably, Algorand~\cite{gilad2017algorand} uses VRFs (verifiable random functions)~\cite{micali1999verifiable,dodis2005verifiable} to realize a lottery every \epoch. (While there exist more than one version of Algorand's private lottery algorithm, we consider its first version, as they do not differ fundamentally.) Quite naturally, one's private input to $VRF_{sk}(\cdot)$ is its secret key. The lottery is given by
\[
VRF_{sk}(\drboutput_{\epochv - 1} \mathbin\Vert role) < target
\]
where $role$ is some parameter specific to Algorand. As both $\drboutput_{\epochv - 1}$ and $role$ are already public and ungrindable at the time of computation, Algorand makes sure $sk$ is likewise ungrindable by requiring that $sk$ is committed in advance. Similar are private lotteries for protocols like Ouroboros Praos~\cite{david2018ouroboros}, Caucus~\cite{azouvi2018winning} (where a hash chain replaces VRFs), and NV (from Nguyen-Van et al.~\cite{nguyen2019scalable}). See Table~\ref{table:committee-based} for details.

% \noindent\textbf{Replacing VRF with $H(\cdot)$ and hash chain.} In Caucus, the VRF is replaced by a hash function and a hash chain, i.e. a list ($h_1, ..., h_m$) with $h_\epochv = H(h_{\epochv + 1})$ for all $\epochv = 1, ..., m - 1$ where $h_m = s$ for some random seed. A hash chain provides the functionality of provably committing to private inputs as one can publicize one $h_\epochv$ at a time (i.e. $h_\epochv$ in \epoch $\epochv$) such that doing so commits to $h_{\epochv + 1}$. Thus, each participant of Caucus independently generates a private hash chain comprising $m$ private inputs such that the lottery is given by
% \[
% H(h_\epochv \oplus \drboutput_{\epochv - 1}) < target
% \]
% which simply involves a hash function. One downside is that the hash chain needs to be periodically regenerated, as $m$ is finite while we desire our DRB to run indefinitely. Verifying a winning ticket in Caucus also requires work linear in the length of the hash chain (though a tree-based structure could make this cost logarithmic).\\

Private lotteries provide two notable benefits: resilience to DoS attack (due to its property of delayed unpredictability~\cite{azouvi2018winning} where one cannot predict the eligibility of honest nodes until they reveal) and \textit{independent participation} (i.e. nodes do not have to know other participants in advance to participate) allowing less communication cost as well as a more permissionless setting. Nonetheless, it can introduce the possibility of biasing via withholding (as discussed in Section~\ref{subsection:withholding}).

\subsection{Step 2. Beacon Output Generation}
\label{subsection:beacon-output-generation}
Given a concrete committee $\committee_\epochv$, the next step is to output $\drboutput_\epochv$. While a typical commit-reveal-recover run among nodes in $\committee_\epochv$ may be sufficient to realize a DRB, other approaches provide different tradeoffs. Largely, we classify these variations into two: one that requires \textit{fresh} (independently generated on the spot) per-node entropy (contribution) and one that combines previous beacon output with \textit{precommitted} (independently generated but precommitted, hence ungrindable) per-node entropy.

\subsubsection{Fresh Per-Node Entropy}
\label{subsubsection:fresh}
Beacon output generation process involving fresh (also referred to as true randomness~\cite{cascudomt, das2021spurt} as opposed to pseudorandomness) per-node entropy for committee-based protocols is typically a commit-reveal-recover variant from Section~\ref{section:commit-reveal-recover}. Some protocols in this family include the following.\\

\noindent\textbf{Share-Reconstruct-Aggregate.} In Ouroboros, nodes in $\committee_\epochv$ (i.e. slot leaders of epoch $\epochv$) perform a RandShare-style share-reconstruct-aggregate using PVSS to output $\drboutput_\epochv$. RandHound uses a similar approach, facilitated by an \epoch leader.\\

\noindent\textbf{Share-Aggregate-Reconstruct.} In SPURT, OptRand, and BRandPiper, nodes in $\committee_\epochv$ perform a SecRand-style share-aggregate-reconstruct to output $\drboutput_\epochv$. BRandPiper has a twist, however: it utilizes the idea of buffering PVSS shares in advance. While there exists one entropy provider per \epoch, $n$ secrets (one from each node) are combined such that it provides the ideal 1-unpredictability property as opposed to $t$-unpredictability (as in HydRand or GRandPiper). The trick is that each \epoch leader generates $n$ fresh secrets that become combined with others' secrets in the next $n$ \epochs, respectively. One node distributes $O(n^2)$ PVSS shares (buffered by other nodes) per \epoch in BRandPiper whereas, in a typical share-aggregate-reconstruct like SPURT and OptRand, each of $O(n)$ nodes distributes $O(n)$ PVSS shares per \epoch (with no buffering).\\

\noindent\textbf{From Threshold Encryption.} Similar to HERB, entropy providers in NV~\cite{nguyen2019scalable} contribute their fresh entropy using ElGamal although they use its classical, non-threshold version due to NV's centralized model in which a third party called the Requester is the direct recipient of a beacon output. As a result, each entropy provider generates and encrypts its entropy and sends it to the Requester, which then decrypts all the messages received from entropy providers and outputs their sum as $\drboutput_\epochv$. Naturally, this Requester version of NV can be modified into what we call \textit{NV++}, which differs from NV in two ways. First, nodes in $\committee_\epochv$ (once finalized) can be made to perform HERB among themselves. This eliminates the existence of the centralized Requester. Second, entropy provision (i.e. broadcasting one's entropy) can be coupled with proof of membership to $\committee_\epochv$ (i.e. broadcasting the fact that a node has won the VRF private lottery). In NV, these two are separate steps potentially incurring adaptive insecurity (a concept delineated in Section~\ref{subsection:adaptive}). Thus, NV++ distributes the Requester and achieves adaptive security.

\subsubsection{Combining Previous Output and Precommitted Per-Node Entropy}
\label{subsubsection:precommitted}
To optimize and lower communication cost, we can require generally less input from entropy providers each \epoch. The canonical optimization involves utilizing $\drboutput_{\epochv - 1}$ as a source of entropy to produce $\drboutput_\epochv$. Nonetheless, the caveat in doing so is that grinding attack may become a possibility once $\drboutput_{\epochv - 1}$ becomes public, which is why it is necessary to require entropy providers' contribution for \epoch $\epochv$ to be precommitted before combining with $\drboutput_{\epochv - 1}$ to output $\drboutput_\epochv$. This prevents grindability while taking advantage of the convenience of $\drboutput_{\epochv - 1}$. Such a requirement can be observed in many committee-based protocols, even though their details seem unrelated on the surface.
\begin{itemize}
\item HydRand and GRandPiper. Each \epoch, an entropy provider (i.e. \epoch leader) in HydRand commits its entropy that becomes opened in the next \epoch it is selected as the leader again. In other words, the \epoch leader's precommitted entropy $e_{\tilde{\epochv}}$ from its last \epoch $\tilde{\epochv}$ of leadership is the one that becomes combined with $\drboutput_{\epochv - 1}$ in the form of $h^{e_{\tilde{\epochv}}}$ to generate
\[
\drboutput_\epochv = H(\drboutput_{\epochv - 1} \mathbin\Vert h^{e_{\tilde{\epochv}}})
\]
while PVSS recovery is used in case the leader fails to open $e_{\tilde{\epochv}}$ in \epoch $\epochv$. Notable in HydRand is the fact (achieving ungrindability of $h^{e_{\tilde{\epochv}}}$) that one honest node must be present in any $t + 1$ consecutive \epochs due to the requirement that a leader cannot gain another leadership in the next $t$ \epochs. Similar overall is GRandPiper's beacon output generation (see Table~\ref{table:committee-based}).
\item Algorand and Ouroboros Praos. These schemes use a VRF for beacon output generation (rather than only for committee selection as in NV++). The secret key $sk$ of the \epoch leader often corresponds to precommitted per-node entropy as long as the assumption that nodes cannot switch their $sk$ at the time of VRF's computation holds. Algorand's beacon output is therefore given by
\[
\drboutput_\epochv = VRF_{sk}(\drboutput_{\epochv - 1} \mathbin\Vert \epochv)
\]
combining the previous output $\drboutput_{\epochv - 1}$ with the precommitted entropy $sk$. Note that the input to the VRF in beacon output generation is different from that in committee selection, as the VRF output in committee selection is always going to be less than $target$ by design. Ouroboros Praos' beacon output is generated similarly (see Table~\ref{table:committee-based}).
\item Caucus. Each new reveal ($h_\epochv$ in \epoch $\epochv$) from an entropy provider's private hash chain in Caucus corresponds to that node's precommitted entropy. The beacon output is given by
\[
\drboutput_\epochv = h_\epochv \oplus \drboutput_{\epochv - 1}
\]
such that it naturally follows its committee selection mechanism $H(h_\epochv \oplus \drboutput_{\epochv - 1}) < target$.
\end{itemize}
