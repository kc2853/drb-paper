%!TEX root = main.tex

\section{Concluding Remarks}
\label{section:conclusion}
%In this paper, we systematize distributed randomness beacons.
%and describe a modular framework comprising two components: selection of entropy providers and beacon output generation, which can be used to analyze a DRB's design.
Our systematization highlights important insights both for practitioners and researchers. 
%Based on practical considerations such as scalability (in $n$), flexibility (reconfiguration and independent participation), robustness (fault tolerance and max damage), and randomness quality (true randomness versus pseudorandomness),
Based on our comparative framework, we would advise practitioners planning to deploy a DRB to consider the following high-level guidelines:
\begin{itemize}
    \item Delay-based protocols stand above the competition in terms of scalability, flexibility, and robustness, enabling an efficient DRB with unlimited, open participation and security given any honest participant. In theory, VDFs appear to be a silver bullet for DRBs, though they have yet to be widely used in practice and assumptions about VDF security and hardware speeds remain relatively new. They also invoke a unique practical cost in that \emph{somebody} must compute a VDF (preferably by running specialized hardware), which also induces latency into the DRB.
    \item If not using VDFs, practitioners need to think critically about two design dimensions: how large is the set of participants, and how frequently will it change? Given a small, static set of participants, DKG-based protocols, e.g. HERB (from threshold encryption) and drand (from DVRF), scale better than PVSS-based protocols. HERB and drand are both competitive in this setting, differing in randomness quality and max damage.
    \item For a small but dynamic set of participants, PVSS-based protocols offer better flexibility (by avoiding a costly DKG setup per reconfiguration) and randomness quality. Committees may be needed to scale to more participants.
    \item Given a large, dynamic set of participants, protocols with private lotteries like Algorand offer better scalability and flexibility simultaneously although the randomness quality is potentially affected by withholding.
    \item Finally, escrow-based protocols are suitable against purely financially-motivated adversaries in applications such as lotteries or finance, at the cost of locking up some amount of capital during the protocol.
\end{itemize}

We conclude by identifying the following areas which we consider most promising for further research.
\begin{itemize}
    \item While VDFs are a promising tool, practical deployment requires good estimates of the lower bound of wall-clock VDF evaluation time. More research is needed to gain confidence in the security of underlying VDF primitives (such as repeated modular squaring), and hardware implementations must be built to provide practical assurance.
    \item VDFs might be useful as a modular layer in strengthening other DRBs in a ``belt-and-suspenders'' approach, though this does not appear to have been explored yet.
    % \item To avoid the need for VDF evaluation (and latency) in the optimistic case, timed commitments (a related primitive to VDFs) might be used instead as yet another mechanism to recover from aborts in commit-reveal.
    \item Vulnerable to withholding, protocols based on private lotteries can generate biased outputs. Though the guarantee of a single lottery winner every \epoch via SSLE makes withholding detectable, extending these protocols to enable tracing of which node withheld is a promising direction for further research.
    \item With the exception of VDF-based protocols like Unicorn++, all other DRBs assume a permissioned setting requiring some initial setup (e.g. PKI or DKG) to establish participants' identities. It is an open question which non-VDF-based protocols can be extended to enable ad hoc, permissionless participation.
    \item Most existing DRBs assume synchronous communication, which may fail in practice. Extending protocols to handle asynchrony is an important challenge.
    \item Most papers today use game-based security definitions. Universal Composability (UC) security proofs~\cite{canetti2001universally} could be a useful tool for proving more robust and modular security results.
    %\item Key subroutines such as DKG or Byzantine consensus can be studied and optimized further for better communication and computational efficiency, flexibility, and robustness.
    \item Finally, there is a gap between the systems-based literature on DRBs and the traditional cryptographic literature on randomness extractors~\cite{trevisan2000extracting,trevisan2001extractors}, with DRBs simply assuming cryptographic primitives such as hash functions work as extractors in practice. Utilizing the existing theory of extractors could prove useful in scenarios where high-quality DRB outputs are required directly.
    % proactively detecting corruption and automatic recovery
    % \item Multiple independent protocols might be combined in a modular way to build a heterogeneous beacon. Mt. Random~\cite{cascudo2021mt} explores this idea by building a multi-tiered beacon with PVSS-based (Scrape variant), DVRF-based (drand variant), and VRF-based (Ouroboros Praos variant) protocols, with each tier providing different trade-offs between cost and randomness quality. Extending this framework of juxtaposition to combine other primitives can be explored further.
\end{itemize}
